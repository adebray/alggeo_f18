One of the advantages of algebraic geometry is the ability to work relative to a given space, which generalizes
choosing a base field (or ring).
\begin{defn}
Let $S$ be a space. A \term{scheme over $S$} is a space $X$ with a map $X\to S$, often just written $X/S$, such
that for all affine schemes $T$ and maps $T\to S$, $X\times_S T$ is a scheme. A morphism of schemes over $S$ is a
morphism of schemes which commutes with the two maps to $S$.
\end{defn}
In the same way one can define affine schemes over $S$. If $S = \Spec A$, for $A$ a ring, we might write $X/A$
instead of $X/S$ and say schemes over $A$; often $A$ will be a field.

We defined spaces as functors $\cat{CommRing}\to\Set$, and there's a similar description for schemes over $A$.
\begin{prop}
Let $A$ be a commutative ring. There's an equivalence of categories between spaces over $A$ and functors
$\cat{CommAlg}_A\to\Set$ (where we ignore the zero algebra).
\end{prop}
\begin{proof}[Proof sketch]
Given $X\colon\cat{CommAlg}_A\to\Set$, we can define a functor on all commutative rings by sending $B$ to the set
of pairs of $(i,x)$ where $i\colon A\to B$ is an $A$-algebra structure on $B$ and $x\in X(B)$. Then the forgetful
map $(i,x)\mapsto i$ defines the desired map to $\Spec A$.

In the other direction, let $p\colon X\to\Spec A$ be a scheme over $A$. We'll define a functor on commutative
$A$-algebras by sending $(B, i\colon A\to B)$ to the set of maps $\vp\colon \Spec B\to X$ for which the diagram
\begin{equation}
\gathxy{
	& X\ar[d]^p\\
	\Spec B\ar[r]^{i^*}\ar[ur]^\vp & \Spec A
}
\end{equation}
commutes.
\end{proof}
\begin{exm}
Complex conjugation is $\Z$-linear (and even $\R$-linear) but not $\C$-linear, and therefore induces a map of
schemes $\Spec\C\to\Spec\C$ which is a map of schemes over $\R$, but not of schemes over $\C$.
\end{exm}
\begin{prop}
\label{pullbackscheme}
Let $X$, $Y$, and $Z$ be schemes together with maps $X\to Z$ and $Y\to Z$. Then $X\times_Z Y$ is a scheme.
\end{prop}
\begin{proof}
If $X = \Spec A$, $Y = \Spec B$, and $Z = \Spec C$ are affine, this is certainly true: the pullback is $\Spec
A\otimes_C B$. Now we'll show more general cases reduce to this one.

If $Y$ and $Z$ are affine but $X$ isn't, then $X$ admits an open cover $\fU$ by affines, and $\set{U\times_Z Y\mid
U\in\fU}$ is an affne open cover of $X\times_Z Y$. In the same way, we may assume only that $X$ and $Z$ are affine.

Therefore if you only assume $Z$ is affine, you can pick affine open covers of $X$ and $Y$ called $\fU$ and
$\mathfrak V$, respectively. Then $\set{U\times_Z V\mid U\in\fU, V\in\mathfrak V}$ is an affine open cover of
$X\times_Z Y$.

Next, we assume $X$ and $Y$ are affine, but $Z$ might not be.\footnote{From here, the proof was finished up in
Friday's lecture.} Let $\mathfrak W$ be an affine open cover of $Z$, and $W\in\mathfrak W$. By definition, the map
\begin{equation}
	X\times_Z W\longrightarrow X
\end{equation}
is an open embedding, and this implies that $X\times_Z W$ is a scheme (we called these quasi-affine): it's the
complement of a closed embedding $\Spec A/I\to X = \Spec A$, and is covered by $\set{D(f)}$ where $\set f$
generates $I$. Anyways, then $X\times_Z Y$ is covered by
\begin{equation}
	\mathfrak W'\coloneqq \set{(X\times_Z W)\times_W (Y\times_Z W)\mid W\in\mathfrak W}.
\end{equation}
Since $W$ is affine, this is a scheme by one of the earlier cases. Therefore $X\times_Z Y$ is covered by schemes,
so it must be a scheme (choose an affine cover of each element of $\mathfrak W'$, and check this is an affine open
cover of $X\times_Z Y$).

Finally, we assume none of them are affine. This is the same as the case where $X$ and $Y$ are affine, but now we
can use the previous step to show that if $\fU$ is an open cover of $X$, $\mathfrak V$ is an open cover of $Y$,
$U\in\fU$, and $V\in\mathfrak V$, then $U\times_Z V$ is a scheme.

We've ignored the Zariski sheaf property, but it's relatively simple to show that it's preserved by fiber products.
\end{proof}
\begin{cor}
\label{schemesoverschemes}
If $S$ is a scheme, schemes over $S$ are the same thing as schemes with a map to $S$.
\end{cor}
\begin{proof}
We can check the definition on an affine open cover of $S$; \cref{pullbackscheme} tells us that pulling back to $T$
preserves scheminess.
\end{proof}
If $S$ is a space that's not a space, \cref{schemesoverschemes} isn't necessarily true.
\subsection*{Quasicoherent sheaves and/or linear algebra.}
In commutative algebra, one often studies a ring by studying its modules; these are linear-algebraic in nature,
which can make them easier to reason about. The analogue for schemes is quasicoherent sheaves.
\begin{defn}
Let $X$ be a scheme. A \term{quasicoherent sheaf} (QC sheaf) $\sF$ on $X$ is data of, for all maps $f\colon \Spec
A\to X$, an $A$-module $\sF_f$, and for every map $g\colon\Spec B\to\Spec A$, an isomorphism
\begin{equation}
	\alpha_{f,g}\colon \sF_{g\circ f} \overset\cong\to \sF_f\otimes_A B
\end{equation}
of $B$-modules, and such that a cocycle condition holds: given a triple
\begin{equation}
\xymatrix{
	\Spec C\ar[r]^-h & \Spec B\ar[r]^-g & \Spec A\ar[r]^-f &X,
}
\end{equation}
$\alpha_{f, g\circ h} = \alpha_{f\circ g, h}$ as maps $\sF_{f\circ g\circ h}\cong  (\sF_f\otimes_A B)\otimes_B C$,
using the natural isomorphism $(\sF_f\otimes_A B)\otimes_B C\cong \sF_f\otimes_A C$.\footnote{The cocycle condition
can be expressed more concisely by asking that $\sF$ is a functor from the category of affine schemes to abelian
groups.} A morphism of quasicoherent sheaves $\sF\to\sG$ is data of maps of $A$-modules $\sF_f\to\sG_f$ for all
$f\colon\Spec A\to X$, such that all induced diagrams commute. The category of QC sheaves on $X$ is denoted
$\QCoh(X)$.
\end{defn}
\begin{rem}
The word ``quasicoherent'' isn't really great unless you're playing Scrabble. It grew out of a generalization of
coherent sheaves, which originally came from the analytic setting, where the name was more reasonable. You should
think of analogues of modules when you hear QC sheaves.
\end{rem}
This is a lot of data! So we're going to find a way to express a quasicoherent sheaf with less data.
\begin{prop}
\label{QCequiv}
If $X = \Spec A$, the functor $\Gamma\colon\QCoh(X)\to\Mod_A$ sending $\sF\mapsto\sF_\id$ is an equivalence of
categories, with inverse sending an $A$-module $M$ to the sheaf $\sF_M$ defined by $(\sF_M)_f\coloneqq M\otimes_A
B$ for all $f\colon\Spec B\to X$.
\end{prop}
\begin{exm}
For any scheme $X$, there's a quasicoherent sheaf $\sO_X$, called the \term{structure sheaf} of $X$, defined to
send $f\colon\Spec A\to X$ to $(\sO_X)_f = A$. The maps are what you think they are.
\end{exm}
