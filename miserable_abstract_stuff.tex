The goal of today's lecture is to define a scheme, first heuristically and then rigorously.
\begin{quotdefn}
\label{firstschemedefn}
A scheme is a ``space'' that is a Zariski sheaf which admits an ``open cover'' by affine schemes.
\end{quotdefn}
Of course, in order to do this, we need to know what all of these words --- spaces, Zariski sheaves, affine
schemes, and open covers --- mean in this setting.
\begin{rem}
There's another approach to schemes using the formalism of \term{locally ringed spaces}, which is followed by
Hartshorne, Vakil, and many others. It's more concrete, but it makes it harder to think about what a specific
scheme, such as projective space, is supposed to be.
\end{rem}
The motivation for \cref{firstschemedefn} is that a scheme should be something which is locally defined by
algebraic equations. For example, let's look at the \term{Fermat equation} $X_n = \set{x^n+y^n = z^n}$. Fermat was
interested in solutions in $\Z$, but the set of solutions makes sense in any commutative ring. This suggests our
definition of space, which is not the same as a topological space.
\begin{defn}
A \term{space} is a functor $X\colon\cat{CommRing}\to\Set$.
\end{defn}
Concretely, this means that for every ring $A$, we get a set $X(A)$, and for every map of commutative rings
$f\colon A\to B$, we get a map of sets $X(f)\colon X(A)\to X(B)$, and these morphisms should compose well (meaning
that $X(f\circ g) = X(f)\circ X(g)$ and $X(\id) = \id$). For example, we could let $X_n(A)$ denote the set of
solutions to the Fermat equation in the ring $A$; then, if we've solved it in $A$, we can map the solution into $B$
via $f\colon A\to B$, and we'll obtain a solution in $B$, so this defines a space $X_n$.

We should also say how spaces interact.
\begin{defn}
A \term{morphism of spaces} $f\colon X\to Y$ is data of, for all commutative rings $A$, a map $f_A\colon X(A)\to
Y(A)$ such that for all ring homomorphisms $g\colon A\to B$, the diagram
\[\xymatrix{
	X(A)\ar[r]^{f_A}\ar[d]^{X(g)} & Y(A)\ar[d]^{Y(g)}\\
	X(B)\ar[r]^{f_B} & Y(B)
}\]
commutes.
\end{defn}
Schemes are special examples of spaces, in a way that feels surprisingly down-to-Earth.

Our first example of a space is the solutions to the Fermat equation in $A$, as discussed above. Here's another
example.
\begin{exm}
Let $A$ be a commutative ring. We'll define the space $\Spec A$ to be the functor $(\Spec A)(B) = \Hom(A, B)$;
given a ring homomorphism $\vp\colon B\to C$, we use the map $\Hom(A,B)\to\Hom(A,C)$ given by postcomposition with
$\vp$.
\end{exm}
\begin{defn}
An \term{affine scheme} is a space of the form $\Spec A$ for some $A$.
\end{defn}
You don't have to be a commutative algebra expert to learn algebraic geometry, but you can see that commutative
algebra is built into the definitions of algebraic geometry, so some commutative algebra knowledge is helpful.
\begin{exm}
The space $X_n$ sending $A$ to the solutions of the Fermat equation in $A$ is an affine scheme; explicitly,
\[X_n\cong \Spec \Z[x,y,z]/(x^n+y^n-z^n).\]
This is because a ring homomorphism $\Z[x,y,z]/(x^n+y^n-z^n)\to A$ is exactly the data of $x,y,z\in A$ satisfying
the relation $x^n+y^n-z^n = 0$.
\end{exm}
\begin{lem}[Yoneda lemma]
For all spaces $X$, $\Hom_{\cat{Spaces}}(\Spec A, X)\cong X(A)$.
\end{lem}
\begin{proof}[Proof sketch]
First we define a map from $\Hom_{\cat{Spaces}}(\Spec A, X)$ to $X(A)$. Specifically, a map $f\colon\Spec A\to X$
is the data of for all commutative rings $B$, $\Spec(A)(B)\to X(B)$. Take $B = A$; then, $\Spec(A)(A) = \Hom(A)$,
so take the image of the identity. It remains to check this is an equivalence.
\end{proof}
\begin{cor}
$\Hom_{\cat{Spaces}}(\Spec A, \Spec B)\cong\Hom_{\cat{CommRing}}(B, A)$.
\end{cor}
It's interesting that the direction reverses!
\begin{proof}
By the Yoneda lemma, $\Hom_{\cat{Spaces}}(\Spec A, \Spec B) = \Spec(B)(A) = \Hom(B,A)$.
\end{proof}
This tells you that as long as you make sure to reverse the arrows, anything you can do with commutative rings, you
can do with affine schemes, and vice versa.
\subsection*{Fiber products.}
This is a categorical construction which we're going to use a lot.
\begin{defn}
Let $X$, $Y$, and $Z$ be sets and $f\colon X\to Z$ and $g\colon Y\to Z$ be set maps. Then the \term{fiber product}
of $X$ and $Y$ over $Z$ is
\begin{equation}
	X\times_Z Y\coloneqq\set{(x,y)\in X\times Y\mid f(x) =g(y)}.
\end{equation}
If $X$, $Y$, and $Z$ are spaces, and $f$ and $g$ are maps of spaces, then the \term{fiber product} of $X$ and $Y$
over $Z$ is the space defined by
\begin{equation}
	(X\times_Z Y)(A)\coloneqq X(A)\times_{Z(A)} Y(A).
\end{equation}
\end{defn}
Technically, the notation should include $f$ and $g$, but in practice there's usually no ambiguity.
\begin{exm}
Suppose we're given commutative rings $A$, $B$, and $C$ and maps $\Spec B\to\Spec C$ and $\Spec A\to\Spec C$ (which
are equivalent data to maps $\vp\colon C\to A$ and $\psi\colon C\to B$). Then
\[\Spec A\times_{\Spec C}\Spec B \cong \Spec (A\otimes_C B),\]
where $C$ acts on $A$, resp.\ $B$, through $\vp$, resp.\ $\psi$. It's worth working through this one on your own,
though it's not extremely hard.
\end{exm}
We'll define some properties of affine schemes with geometric names, but the definitions will rest on algebraic
properties of rings. One of the real miracles of algebraic geometry is that this really works to define geometry,
and even extends geometric intuition to places such as finite fields that are otherwise very hard to reason about.
\begin{defn}
\label{closedemb}
A morphism $\Spec B\to\Spec A$ is a \term{closed embedding} if the induced map $A\to B$ is surjective.
\end{defn}
Equivalently, $B = A/I$ for some ideal $I$ of $A$.

The geometric idea behind defining $\Spec A$ is that geometric objects have a ring of functions on them, e.g.\ a
smooth manifold $M$ has a ring $C^\infty(M)$ of smooth $\R$-valued functions, and a map of manifolds $M\to N$
induces a map in the other direction by pullback: $C^\infty(N)\to C^\infty(M)$. Functional analysis results such as
the Gelfand-Naimark theorem tell you what data you need to add to $C^\infty(M)$ to recover $M$ as a topological
space, and we're trying to imitate this in a more abstract algebraic setting.

This context allows us to explain why \cref{closedemb} deserves to be called a closed embedding: let $I =
(f_1,f_2,\dotsc)$, so
\begin{equation}
	\Spec A/I = \set{f_i = 0\text{ for all } i} = \set{f = 0\text{ for all } f\in I}.
\end{equation}
So we think of $\Spec B$ as some kind of closed subspace of $\Spec A$, and $I$ as the ideal of functions on $\Spec
A$ which vanish on $\Spec B$. This intuition can be turned into something precise.

Using fiber products, we can extend this to all spaces.
\begin{defn}
\label{closedembsp}
A map $X\to Y$ of spaces is a \term{closed embedding} if for all maps $\Spec A\to Y$, the ``pullback'' $\vp$ in the
fiber product diagram
\begin{equation}
\label{fibclos}
\gathxy{
	\Spec A\times_Y X\ar[r]^-\vp\ar[d] & \Spec A\ar[d]\\
	X\ar[r] & Y
}
\end{equation}
is a closed embedding of affine schemes. In particular, we require $\Spec A\times_Y X$ to be an affine scheme,
which is not always satisfied.
\end{defn}
For a quick consistency check, we should ask that \cref{closedemb,closedembsp} agree on affine schemes, and indeed,
if $I\subset A$ is an ideal, and $\Spec B\to\Spec A$ is a closed embedding in the sense of \cref{closedemb},
then~\eqref{fibclos} looks like
\begin{equation}
\gathxy{
	\Spec(A/I\otimes_A B)\ar[r]\ar[d] & \Spec B\ar[d]\\
	\Spec(A/I)\ar[r] & \Spec A,
}
\end{equation}
and since $A/I\otimes_A B\cong B/BI$, this is a closed embedding in the more general sense as well.

We'd also like to know what an open embedding is. We'd like to say that it's something whose complement is a closed
embedding. Let's make this precise.
\begin{defn}
Let $Z\inj X$ be a closed embedding of spaces. The \term{complement} $X\setminus Z$ of $Z$ in $X$ is the space with
$(X\setminus Z)(A)$ the set of $x\in X(A) = \Hom_{\cat{Spaces}}(\Spec A, X)$ such that the diagram
\begin{equation}
\label{emptyfib}
\gathxy{
	\varnothing\ar[r]\ar[d] & Z\ar[d]\\
	\Spec A\ar[r]^-x & X
}
\end{equation}
is a fiber product diagram. Here $\varnothing = \Spec(0)$, which sends every ring to the empty
set.\footnote{Caution: this is only true if we work with functors on nonzero rings. However, $\varnothing = \Spec
0$ still counts as affine. There are other ways to correct this issue, but this is among the fastest and cheapest.}
\end{defn}
\begin{defn}
If $X = \Spec A$ is an affine scheme, an \term{open embedding} is a map of spaces $j\colon U\to X$ such that $U =
X\setminus Z$ for some closed embedding $Z\inj X$.
\end{defn}
\begin{exm}
Letting $X = \Spec A$, if $f\in A$ and $Z = \Spec(A/f)$, the map $A\surj A/f$ induces a closed embedding $Z\inj X$.
Its complement is $\Spec A[f^{-1}]$, the \term{localization} of $A$ at $f$, so $\Spec A[f^{-1}]\to\Spec A$ is an
open embedding.
\end{exm}
The intuition is that $f$ generates the ideal of functions that vanish precisely on the closed subset $Z$.
Therefore on the complement of $Z$, they should be invertible, so we adjoin an inverse to $f$.
\begin{lem}
\label{factors_through}
Let $X = \Spec A$ and $Z = \Spec A/I$. Then maps $\Spec B\to X\setminus Z$ correspond bijectively to maps $A\to B$
such that $B\cdot I = B$.
\end{lem}
\begin{proof}
The diagram~\eqref{emptyfib} specializes to
\begin{equation}
\gathxy{
	\varnothing\ar[r]\ar[d] & \Spec(A/I)\ar[d]\\
	\Spec B\ar[r] & \Spec A,
}
\end{equation}
and this fiber product is $\Spec(B\otimes_A A/I) = \Spec(B/IB) = \varnothing$, which is equivalent to $IB = B$.
\end{proof}
\begin{exm}
\term{Affine $n$-space} over $\Z$ is the affine scheme $\A_\Z^n\coloneqq\Spec \Z[x_1,\dotsc,x_n]$, and
$0\inj\A_\Z^n$ is the closed embedding corresponding to the ideal $(x_1,\dotsc,x_n)$. The complement
$\A_\Z^n\setminus 0$ is not affine for $n > 1$! We'll prove that later when we have more tools.
\end{exm}
\begin{ex}
Show that $\A_\Z^n\setminus 0$ is the space which maps a ring $A$ to the set of $n$-tuples $(x_1,\dotsc,x_n)\in
A^n$ such that the equation $\sum x_iy_i = 1$ has a solution.
\end{ex}
