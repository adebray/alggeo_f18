Today, we're going to continue proving \cref{ASisS}, that affine schemes are schemes. We're still working on the
special case that $\A^1$ is a scheme; the key piece of the proof is showing that it's a Zariski sheaf.
\begin{defn}
Let $S$ be a space and $\fU$ be an open cover of $S$. A \term{refinement} of $\fU$ is an open covering $\mathfrak
V$ of $S$ such that for all $U\in\fU$, $\mathfrak V_U\coloneqq \set{V\in\mathfrak V\mid V\subset U}$ is an open
covering of $U$.
\end{defn}
The Zariski sheaf condition for maps $X\to S$ is a constraint on compatible functions on all open covers of $S$. If
we only ask about a specific open cover $\fU$, we say ``the Zariski sheaf property for $X$ with respect to $\fU$.''
\begin{lem}
\label{refinesheaf}
Let $X$ and $S$ be spaces, $\fU$ be an open cover of $S$, and $\mathfrak V$ be a refinement of $\fU$. Suppose the
Zariski sheaf property holds for $X$ with respect to $\mathfrak V$, and for each $U\in\fU$ with respect to
$\mathfrak V_U$, then it holds with respect to $\fU$.
\end{lem}
After you unwind all the definitions, this is a definition check which isn't very hard.
\begin{rem}
One corollary of \cref{refinesheaf} is that in the definition of the sheaf property, we may replace ``for all
affine schemes $S$'' with ``for all spaces $S$.'' All of the definitions were built from the beginning to favor
affine schemes as important or special, and this is one consequence.
\end{rem}
\begin{defn}
A \term{big basic open covering} of an affine scheme $S$ is an open covering by sets of the form $D(f_i)$ as in
\cref{basiclem}, but over a possibly infinite indexing set.
\end{defn}
This is only a temporary definition. The Zariski sheaf property for $X$ and every basic open
covering of an affine scheme $S$ implies the Zariski sheaf for all big basic open coverings.
\begin{prop}
\label{bigprop}
Let $S$ be an affine scheme and $\fU$ be an open cover of $S$. Then there's a big basic open covering of $S$ refining
$\fU$.
\end{prop}
\begin{proof}
Write $S = \Spec A$ and for each $U\in\fU$, let $Z_U\coloneqq S\setminus U$; the inclusion $Z_U\inj S$ is a closed
embedding, so $Z_U = \Spec(A/I_U)$ for some ideal $I_U\subset A$. Recall from \cref{qc} that since $\fU$ is an open
covering,
\begin{equation}
	\sum_{U\in\fU} I_U = A,
\end{equation}
and this is an equivalent condition. Consider the big basic open cover
\begin{equation}
	\mathfrak V \coloneqq \set{D(f)\mid f\in I_U\setminus 0\text{ for some } U\in\fU}.
\end{equation}
That this is a big basic open cover is because an ideal is generated by its elements. It's also a refinement of
$\fU$, which follows from a more general lemma.
\begin{lem}
Let $U = S\setminus Z$, where $S = \Spec A$ and $Z = \Spec(A/I)$. Then $\set{D(f)\mid f\in I\setminus 0}$ is an
open cover of $U$.
\end{lem}
\begin{proof}
We want to show that for all $T = \Spec B$ and maps $g\colon T\to U$, the set $\mathfrak V_g\coloneqq
\set{g^{-1}(D(f))\mid f\in I\setminus 0}$ is an open cover of $T$.\footnote{The preimage is defined to be
$g^{-1}(D(f))\coloneqq D(f)\times_U T$.} Then\dots \TODO
\end{proof}
Thus we've proven the proposition.
\end{proof}
\begin{cor}
\label{redtoalg}
Let $S$ be an affine scheme with an open covering $\fU$. Then there's a big basic open covering $\mathfrak V$
refining $\fU$ and with the property that for all $U\in\fU$, $\set{V\in\mathfrak V\mid V\subset U}$ is a big basic
open covering of $U$.
\end{cor}
This is the technical proposition that lets us reduce to algebra.
\begin{rem}
\label{bigbasic}
\Cref{redtoalg} also tells us that a big basic open covering of a space $X$ is an open covering $\fU$ of $X$ such
that for all maps of affine schemes to $X$, the pullback of $\fU$ is also a big basic open covering.
\end{rem}
\begin{cor}
$\A^1$ is a Zariski sheaf.
\end{cor}
\begin{proof}
We showed that $\A^1$ is a Zariski sheaf with respect to all basic open covers of affine schemes, hence for all big
basic open covers of affine schemes, hence by \cref{bigbasic} with respect to all spaces with big basic open
covers, hence by \cref{bigprop} any affine scheme and any open cover, and therefore any space and any open cover.
\end{proof}
\begin{cor}
Let $I$ be a set and let $\A^I\coloneqq\Spec\Z[\set{x_i\mid i\in I}]$. Then $\A^I$ is a Zariski sheaf.
\end{cor}
\begin{proof}
The sheaf property is preserved under arbitrary products.
\end{proof}
If $I$ is an $n$-element set, then $\A^I$ is also written $\A^n$.
\begin{proof}[Proof sketch of \cref{ASisS}]
We can use this to show that if $X = \Spec A$ is an affine scheme, then it's a Zariski sheaf. Let $I$ be a
generating set for $A$ and $J \subset\Z[\set{x_i\mid i\in I}]$ be the ideal of relations; then, the quotient map
$\Z[\set{x_i\mid i\in I}]\surj A$ defines a closed embedding $X\subseteq \A^I$ cut out by $X = \set{x\mid f(x) =
0\text{ for all } f\in J}$.

One then has to check that the sheaf property is preserved under closed embeddings, which is formal.
\end{proof}
We'll spend the next lecture giving examples of schemes, but here are a few to start with.
\begin{itemize}
	\item As we just showed, affine schemes are schemes.
	\item A \term{quasi-affine scheme} is an open subset of an affine scheme, such as $\A^2\setminus 0$. These are
	indeed schemes (though not always affine): if $U$ is the complement of $\Spec (A/I)\subset A$, then $U$ admits
	a covering by $\set{D(f)\mid f\in I\setminus 0}$.
\end{itemize}
We can use this to prove $\A^2\setminus 0$ isn't affine.
