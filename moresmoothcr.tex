Last time, we stated that over a perfect base field $k$, a curve is smooth iff it's normal iff it's regular.
\begin{lem}
\label{regsming}
If $X$ is finite type over $k$, $k\inj k'$ is a separable field extension, and $x\in X(k')$ is a closed point
corresponding to the maximal ideal $\m_x$, then the map $\m_x/\m_x^2\to x^*\Omega_X^1$ is an isomorphism.
\end{lem}
This map arises in the following way: we have a map $\d\colon\sO_X\to\Omega_X^1\to x^*\Omega_X^1$, and this map
factors through a map $\sO_X/\m_x^2\to x^*\Omega_X^1$; then we precompose with the map
$\m_x/\m_x^2\to\sO_X/\m_x^2$.
\begin{rem}
$\m/\m^2$ is called the \term{Zariski cotangent space}, and is often one's first definition of the cotangent space.
However, the definition $x^*\Omega_X^1$ has better base change properties. Of course, by the lemma, they're
equivalent.
\end{rem}
\begin{proof}[Proof of \cref{regsming}]
The first step is to assume $X = \Spec A$ is affine. Since $k'$ is separable, $A/\m_x^2 = k'\oplus\m_x\oplus\m_x^2$
is a square-zero extension, giving rise to a map $\delta\colon A\to\m_x/\m_x^2$; it's easy to check that $\delta$
is a $k$-linear derivation. This gives rise to a map $\Omega_X^1\to\m_x/\m_x^2$, hence a map
$x^*\Omega_X^1\to\m_x/\m_x^2$, and you can check this is the inverse map.
\end{proof}
Why does the lemma imply our claim about normal/regular implying smooth? Well, if we have a regular curve $X$, a
finite extension $k\inj k'$, and $x\in X(k')$ closed, then $\m_x/\m_x^2$ is one-dimensional; since $k'$ is
separable then $x^*\Omega_X^1$ is one-dimensional, so $X$ is smooth.
\begin{exm}
Separability isn't a very restrictive hypothesis: it holds in characteristic zero, as well as for all finite fields
and all algebraically closed fields. But suppose $k$ is not a perfect field and $\lambda\in k$ isn't a
$p^{\mathrm{th}}$ power. Then, consider the closed point $x = \set{t^p = \lambda}\subset\A_k^1$. If $k'$ is the
residue field of $x$, then $k\inj k'$ is an inseparable field extension.

The maximal ideal corresponding to $x$ is $\m_x = (t^p-\lambda)$. The map $(t^p-\lambda)/(t^p-\lambda)^2\to k'\cdot
dt$ is $\d = 0$, so \cref{regsming} doesn't hold in this setting.
\end{exm}
It is nonetheless useful to consider imperfect fields. ``Experimental evidence'' suggests that if you care about
varieties, you'll probably only define them over perfect fields. But sometimes you'll want to consider families of
varieties parameterized by some data, which can be thought of as a variety over some function field, and this
function field need not be perfect.

Nevertheless, we now assume $k$ is perfect. We now give a construction of smooth projective curves. Start with a
smooth affine curve $X\subseteq\A^n$. Choose one of the coordinate charts $\A^n\subseteq\P^n$; we can thus regard
$X$ as a subscheme of $\P^n$ and take its closure $\overline X$. This is projective, but it might not be smooth.

Thus we will take the \term{normalization} of $\overline X$, which will yield a smooth projective variety.
\begin{rem}
Let $Y = \Spec A$, where $A$ is an integral domain of finite type over $k$. Let $k(Y)$ denote the fraction field of
$A$. If $\overline A$ denotes the integral closure of $A$ in $k(Y)$, let $Y^{\mathrm{nm}}\coloneqq\Spec(\overline
A)$, which is called the \term{normalization} of $Y$.

The normalization has a few nice properties (which we're not going to prove here): $Y^{\mathrm{nm}}$ is finite type
over $k$, and hence the natural map $Y^{\mathrm{nm}}\to Y$ is finite, and an isomorphism on some open subscheme of
$Y'$. Moreover, normalization localizes well on $Y$: if $U\subset Y$ is an affine open, then $U^{\mathrm{nm}} =
Y^{\mathrm{nm}}\times_Y U$. Therefore normalization generalizes to irreducible $k$-schemes of finite type.
\end{rem}
So we can take the normalization $(\overline X)^{\mathrm{nm}}$, which is finite over $\overline X$. It's easy to
see that the composition of a finite morphism then a projective morphism is projective: more generally, using the
Segre embedding, one can show that the composition of two projective morphisms is projective.
\begin{exm}
Assume $\chr(k)\ne 2$ and let $f(t)\in k[t]$ be a separable polynomial. Let $X = \set{y^2 = f(t)}\subseteq\A^2$.
Then $\overline X\subset\P^2$ is not smooth if $\deg f\ge 4$.
\end{exm}
Since we care more about $(\overline X)^{\mathrm{nm}}$ than about $\overline X$, we're going to make the
normalization implicit.
\begin{cor}
If $X$ is a smooth affine curve over $k$, there's a smooth projective curve $\overline X$ and an embedding
$X\inj\overline X$; this data is unique up to unique isomorphism.
\end{cor}
We sometimes call $\overline X$ the \term{compactification} of $X$.
\begin{proof}
Suppose $\overline X_1$ and $\overline X_2$ are both compactifications of $X$. Because $\overline X_1$ is smooth,
it admits a map to $\overline X_2$ (\TODO: I think this is because we have a map $X\to\overline X_2$ and $X$ is a
dense subscheme of $\overline X_1$), and in the same way we have a map $\overline X_2\to\overline X_1$. The
compositions of these maps are the identity on a dense subscheme, hence must be the identity.
\end{proof}
\begin{rem}
Another way to think of $\overline X$ is the initial projective scheme recieving a map from $X$. This follows by
the valuative criterion.
\end{rem}
One reasonable question is to what extent this generalizes.
\begin{defn}
A scheme $S$ is \term{separated} if the diagonal map $\Delta\colon S\to S\times S$ is a closed embedding.
\end{defn}
This is the analogue of the Hausdorff condition in differential topology --- and, just as in differential topology,
the standard counterexample is the line with two origins.
\begin{exm}
The line with two origins is the space $X$ whose functor of points assigns to $\Spec A$ the set of (isomorphism
classes of) open covers $\set{U,V}$ of $\Spec A$ together with maps $f\colon U\to\A^1$ and $g\colon U\to\A^1$ such
that $f$ and $g$ coincide when they're nonzero.

If you figure out what the closed point are, there's one for every closed point of $\A^1$, except there are two
points corresponding to the origin.
\end{exm}
Affine schemes are separated, because the multiplication map $m\colon A\otimes A\to A$ is surjective.
\begin{exm}
Show that $\P^n$ is separated. Or, more generally, any projective scheme (even any quasiprojective
scheme\footnote{A scheme is \term{quasiprojective} if it's an open subscheme of a projective scheme.}) is
surjective. Hint: you'll want to go back to the definition of $\P^n$.
\end{exm}
Separability is a global, not local condition. As such, it can be more subtle than one expects.
\begin{lem}
Suppose $S$ is a separated scheme and $f,g\colon T\rightrightarrows S$. Then the equalizer $\set{f = g}$ is closed
in $T$.
\end{lem}
\begin{proof}
$\set{f = g} = T\times_{S\times S} S$, and the inclusion from this to $T$ is the base change of the diagonal, and
the pullback of a closed morphism is closed.
\end{proof}
\begin{cor}
\label{sepopen}
Suppose $S$ is an irreducible separated scheme with an open cover $\fU$, and $T$ is an affine scheme.\footnote{It might
be possible to remove the affine hypothesis.} Suppose we have a map $S\to T$ such that for each $U\in\fU$, the
induced map $U\to T$ is an open embedding. Then $S\to T$ is an open embedding.
\end{cor}
Again, the standard counterexample is the map from the affine line with two origins to $\A^1$.

Assuming \cref{sepopen}, we can deduce a nice generalization of normalizations.
\begin{cor}
Let $X$ be a separable smooth curve over $k$. Then there's a smooth projective curve $\overline X$ together with an
open embedding $X\inj\overline X$, and this data is unique up to unique isomorphism.
\end{cor}
\begin{proof}
Cover $X$ by nonempty affines $U_i$; then we have embeddings $U_i\inj\overline U_i$; if $U_i$ and $U_j$ intersect,
then $\overline{U_i} = \overline{U_i\cap U_j} = \overline{U_j}$. Therefore $\overline X = \overline U_i$ for any
$i$. All of the map $U_i\to\overline X$ coincide on the intersection, so $X\inj\overline X$ is an open embedding.
\end{proof}
