\label{morecurves}
Let $k$ be a perfect field and $X$ be a smooth curve over $k$. Last time we saw that $X$ is separable iff it's
quasiprojective (i.e.\ an open subscheme of a projective scheme). More specifically, we constructed a smooth
projective curve $\overline X$ and an open embedding $X\inj\overline X$.

Today, we'll discuss morphisms over curves.
\begin{prop}
\label{curvesarefield}
Let $X$ and $Y$ be smooth, projective curves with function fields $k(X)$, resp.\ $k(Y)$. Then the natural map
$\mathrm{Isom}_{\Sch/k}(X,Y)\to \mathrm{Isom}_{\Alg_k}(k(X), k(Y))$ is an isomorphism.
\end{prop}
An isomorphism $k(X)\to k(Y)$ is (equivalent data to) what's called a birational equivalence $X\dashrightarrow Y$;
this says that all birational equivalences of curves come from actual isomorphisms. This fails drastically in
higher dimensions, largely because the valuative criterion doesn't generalize. The associated field is known as the
\term{minimal model program}. It also fails if you remove the hypotheses on smoothness or projectivity.

\begin{proof}
The inverse map is given as follows. Given some isomorphism $\iota \colon k(X) \to k(Y)$, there is a unique map $Y \to X$ which restricts to $\iota$ on the generic point. This can be proven by slightly modifying the proof of the valuative criterion for properness (which related to extending by open sets) to apply to the generic point.
\end{proof}
\begin{prop}
Let $X$ be a separated, smooth curve. Then the map $U\mapsto\Fun(U)$ is an injective map from from nonempty open
affine subschemes of $X$ to integrally closed, infinite-dimensional, finitely generated $k$-subalgebras $A\subseteq
k(X)$ with field of fractions $k(X)$ is injective. If furthermore $X$ is projective, this map is a bijection.
\end{prop}
\begin{proof}
We'll prove the statement for projective curves. Let $A\subseteq k(X)$ and $U\coloneqq\Spec A$. We claim $U$ is a
smooth curve over $k$: it's clearly finite type, and is irreducible, because $A$ is an integral domain. To see that
$U$ is a curve, notice that its field of functions is $k(X)$ again. It's smooth because $k$ is perfect and $A$ is
integrally closed.

Therefore we have a unique smooth projective curve $\overline U$ containing $U$ as an open subscheme. Then
$k(\overline U)\cong k(X)$ canonically, so there's a canonical isomorphism $X\cong\overline U$, so we get an open
embedding $U\inj X$.
\end{proof}
\begin{defn}
Let $X$ and $Y$ be finite-type schemes over $k$. A map $f\colon X\to Y$ is \term{constant} if its scheme-theoretic
image is a closed point of $Y$.
\end{defn}
That is, we want there to be a finite field extension $k\inj k'$ and $f$ to factor through maps $X\to\Spec k'$ and
$\Spec k'\to Y$.\footnote{In fact, the Nullstellensatz implies that $k\inj k'$ is finite.}
\begin{prop}
If $X$ and $Y$ are smooth projective curves, a map $f\colon X\to Y$ is either constant or dominant.
\end{prop}
\begin{proof}
The scheme-theoretic image of $f$ is a closed, irreducible subscheme of $Y$, hence either a closed point or all of
$Y$.
\end{proof}
Furthermore, any nonconstant map of curves induces a map on function fields, which can be seen by checking affine-locally.
%\begin{proof}
%We can assume $Y$ is affine, by replacing it with any affine open; then we can replace $X$ by any affine open. The
%map $X\to Y$ is still dominant, so the induced map $\Fun(Y)\inj\Fun(Y)$ is injective.
%\end{proof}

Now we can generalize \cref{curvesarefield}.
\begin{prop}
\label{nonconst}
Let $X$ and $Y$ be smooth projective curves. The map from the set of nonconstant functions $f\colon X\to Y$ to
$\Hom_{\Alg_k}(k(Y), k(X))$ is an isomorphism.
\end{prop}
The proof is the same, using the valuative criterion.
\begin{thm}
\label{nonconstaff}
Any nonconstant morphism of smooth projective irreducible curves is finite.
\end{thm}
In particular, this says that any nonconstant morphism of smooth projective irreducible curves is affine, which isn't obvious.
\begin{rem}
The projectivity assumption can't be removed--consider the map $\A^1\setminus 0 \to \A^1$. Similarly, the separatedness is also essential--the map from $\P^1$ with the doubled origin to $\P^1$ shows this. 
\end{rem}
Since constant maps are negligible, this says morphism of curves are basically field extensions.
\begin{proof}[Proof of \cref{nonconstaff}]
Given a nonconstant morphism $f\colon X \to Y$ of smooth projective curves, let $U = \Spec A$ be an affine open
subset of $Y$. Then define $B$ to be the integral closure of $A$ in the fraction field $k(X)$. We claim that $\Spec
B = f^{-1}(U)$ and that $\mathrm{Frac}(B) = k(X)$. Assuming these two claims, latter gives that $\Spec B$ is an affine open subset of $X$ and there's a general fact from commutative algebra which implies that $B$ is a finite $A$ module, which shows that $f$ is finite.

To see that $\Spec B = f^{-1}(U)$, note that for any affine open $i\colon \Spec C \subset X$ where $fi$ factors
through $U$, we obtain a map of rings $A \to C$ to an integrally closed subring of $k(X)$. This implies there's a unique map of rings $B \to C$, which implies the claim.

To see that $\mathrm{Frac}(B) = k(X)$, first note that $k(X)/k(Y)$ is a finite extension, which is a fact about
dominant morphisms between irreducible finite type $k$-schemes of the same dimension, which can be proven by
generic finiteness or the ``base times $\A^1$ or finite argument,'' noting that the former would increase dimension.

Now assume $f \in k(X)$. By above, we have an equation $f^n + a_{1}f^{n-1} + \dotsb + a_n = 0$ for $a_i \in k(Y)$.
Since $\mathrm{Frac}(A) = k(Y)$, we can choose a nonzero $g \in A$ for which $ga_i \in A$ for all $i$. But this
implies that $(gf)^n + ga_1(gf)^{n-1} + \dotsb + g^na_n = 0$, so $gf \in B$ (the integral closure of $A$ in
$k(X)$), so $f \in \mathrm{Frac}(B)$.
\end{proof}
