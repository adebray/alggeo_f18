Let $k$ be a perfect field and $X$ be a smooth curve over $k$. Last time we saw that $X$ is separable iff it's
quasiprojective (i.e.\ an open subscheme of a projective scheme). More specifically, we constructed a smooth
projective curve $\overline X$ and an open embedding $X\inj\overline X$.

Today, we'll discuss morphisms over curves.
\begin{prop}
\label{curvesarefield}
Let $X$ and $Y$ be smooth, projective curves with function fields $k(X)$, resp.\ $k(Y)$. Then the natural map
$\mathrm{Isom}_{\Sch/k}(X,Y)\to \mathrm{Isom}_{\Alg_k}(k(X), k(Y))$ is an isomorphism.
\end{prop}
An isomorphism $k(X)\to k(Y)$ is (equivalent data to) what's called a birational equivalence $X\dashrightarrow Y$;
this says that all birational equivalences of curves come from actual isomorphisms. This fails drastically in
higher dimensions, largely because the valuative criterion doesn't generalize.
\begin{proof}
Let's construct an inverse map. Let $\iota\colon k(X)\to k(Y)$ be an isomorphism. By the valuative criterion, \TODO
(I missed this part).

Now, choose $V = \Spec A\subseteq X$ to be an affine open. Then $A\subseteq k(X)\cong k(Y)$. We claim that there's
an open affine $U = \Spec B\subseteq Y$ such that $A\subseteq B$ under this isomorphism. (\TODO: therefore we're
done? I didn't follow)
\end{proof}
\begin{prop}
Let $X$ be a separated, smooth curve. Then the map $U\mapsto\Fun(U)$ is an injective map from from nonempty open
affine subschemes of $X$ to integrally closed, infinite-dimensional, finitely generated $k$-subalgebras $A\subseteq
k(X)$ with field of fractions $k(X)$ is injective. If furthermore $X$ is projective, this map is a bijection.
\end{prop}
\begin{proof}
We'll prove the statement for projective curves. Let $A\subseteq k(X)$ and $U\coloneqq\Spec A$. We claim $U$ is a
smooth curve over $k$: it's clearly finite type, and is irreducible, because $A$ is an integral domain. To see that
$U$ is a curve, notice that its field of functions is $k(X)$ again. It's smooth because $k$ is perfect and $A$ is
integrally closed.

Therefore we have a unique smooth projective curve $\overline U$ containing $U$ as an open subscheme. Then
$k(\overline U)\cong k(X)$ canonically, so there's a canonical isomorphism $X\cong\overline U$, so we get an open
embedding $U\inj X$.
\end{proof}
\begin{defn}
Let $X$ and $Y$ be finite-type schemes over $k$. A map $f\colon X\to Y$ is \term{constant} if its scheme-theoretic
image is a closed point of $Y$.
\end{defn}
That is, we want there to be a finite field extension $k\inj k'$ and $f$ to factor through maps $X\to\Spec k'$ and
$\Spec k'\to Y$.\footnote{In fact, the Nullstellensatz implies that $k\inj k'$ is finite.}
\begin{prop}
If $X$ and $Y$ are smooth projective curves, a map $f\colon X\to Y$ is either constant or dominant.
\end{prop}
\begin{proof}
The scheme-theoretic image of $f$ is a closed, irreducible subscheme of $Y$, hence either a closed point or all of
$Y$.
\end{proof}
\begin{prop}
\TODO (I missed this)
\end{prop}
\begin{proof}
We can assume $Y$ is affine, by replacing it with any affine open; then we can replace $X$ by any affine open. The
map $X\to Y$ is still dominant, so the induced map $\Fun(Y)\inj\Fun(Y)$ is injective.
\end{proof}
Now we can generalize \cref{curvesarefield}.
\begin{prop}
Let $X$ and $Y$ be smooth projective curves. The map from the set of nonconstant functions $f\colon X\to Y$ to
$\Hom_{\Alg_k}(k(Y), k(X))$ is an isomorphism.
\end{prop}
The proof is the same, using the valuative criterion.
