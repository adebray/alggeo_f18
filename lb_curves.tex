Throughout today's lecture, $X$ is a smooth curve (this means its irreducible components all have dimension 1) over
a field $k$. Let $x = \Spec k'$ be a closed point; we are in the middle of proving that $\sO(-x)$, the ideal sheaf
of $x$, is a line bundle. We finished the proof in the case when $k$ is algebraically closed (in which case $k' =
k$).

In the general case, we constructed a $t\in\sO(-x)$, after possibly replacing $X$ with an open $U\subset X$. The
claim is that there exists a neighborhood $V\subset U$ of $x$ such that multiplication by $t$ is an isomorphism
$\sO_U|_V\overset\cong\to\sO_U(-x)|_V$. This suffices, because $\sO_X(-x)|_{X\setminus x}\cong\sO_{X\setminus x}$.

Recall that we defined $t$ by first choosing an $s\colon U\to\A_k^1$ such that $\Omega_U^1 = \sO_X\cdot\d s$ (in
words, $\d s$ trivializes $\Omega_U^1$); then $s(x)\in\A_k^1$ is a closed point, so here's an irreducible
polynomial $g\colon\A_k^1\to\A_k^1$ with $g(s(x)) = 0$, and we defined $t\coloneqq g\circ s$.

We first claim that $t$ is an epimorphism in a neighborhood of $x$. Consider the diagram
\begin{equation}
\gathxy{
	t^{-1}(0)\ar[r]\ar[d] & U\ar[d]^s\\
	s(x)\ar[r]\ar[d] & \A_k^1\ar[d]^g\\
	0\ar[r] & \A_k^1.
}
\end{equation}
The top square is a pullback square (that's how we defined the preimage), and the bottom square is also a pullback
square. Therefore the outer rectangle is also a pullback square (this is formal), which suffices because \TODO.

Next, we'll show that $t^{-1}(0)$ is smooth over $\Spec k'$, again by abstract nonsense.
\begin{rem}
The definition of $\Omega_{S/k}^1$ makes sense if you replace $k$ by a more general ring, or even in much greater
generality: for any map of schemes $f\colon S\to T$, we can define a sheaf of relative differentials
$\Omega_{S/T}^1$. This has two important properties.
\begin{enumerate}
	\item Given a map $T\to\Spec k$, we obtain an exact sequence
	\begin{equation}
	\label{rightexactdiff}
		\rightexact[\d f][]{f^*\Omega_{T/k}^1}{\Omega^1_{S/k}}{\Omega_{S/T}^1},
	\end{equation}
	which encodes the fact, given a map of rings $\vp\colon B\to A$, that a differential $\delta\colon A\to M$ is a
	$B$-linear iff $\delta(\Im(\vp)) = 0$.
	\item Suppose we have a pullback square
	\begin{equation}
		\gathxy{
			S_2\ar[r]^g\ar[d]^f & S_1\ar[d]^{f_1}\\
			T_2\ar[r] & T_1.
		}
	\end{equation}
	Then there's an isomorphism $g^*(\Omega^1_{S_1/T_1})\cong\Omega^1_{S_2/T_2}$. One can prove this in the affine
	case, by looking at differentials for a pushout of rings.
	\qedhere
\end{enumerate}
\end{rem}
Applying~\eqref{rightexactdiff} to our situation, we have
\begin{equation}
\xymatrix{
	s^*\Omega_{\A^1/k}^1\ar[r]^{-1\mapsto\d s} & \Omega_{U/k}^1\ar[r] & \Omega_{U/\A^1}^1\ar[r] & 0.
}
\end{equation}
Since $\Omega_{U/\A^1}^1 = 0$, then $s^*\Omega_{\A^1/k}^1\cong\Omega_{U/k}^1$. By base change (the second
property), $\Omega_{t^{-1}(0)/k'}^1 = 0$, so $t^{-1}(0)$ is smooth over $k'$.\footnote{\TODO: I might have this
argument slightly out of order, but this is what I think happened.} Therefore $t^{-1}(0)\subseteq U$ is a smooth
closed subscheme, so it's $\Spec$ of a product of fields which are separable over $k'$, or it's a disjoint union of
finitely many distinct closed points. We can then let $V$ be $U$ minus those points, and we're done.

In algebra, $(t) = \m_x$, so that multiplication by $t$ is a map $A\to\m$; geometrically, this means
$\sO_V\to\sO_V(-x)$ is an epimorphism.

We have just one step left: we need to show $t$ isn't a zero divisor, so that $\sO_V\to\sO_V(-x)$ is injective. If
$V = \Spec A$< then $t\in A$ and $(t) = \m$.
\begin{lem}
$\m^n/\m^{n+1}$ is a one-dimensional $A/\m$-vector space generated by $t^n$.
\end{lem}
\begin{proof}
$\m^n$ is clearly generated by $t^n$, so it's at most one-dimensional. Consider the sequence
\begin{equation}
	\xymatrix{
		A/\m\ar[r]^t & \m/\m^2\ar[r]^t & \dotsb
	}
\end{equation}
The only potential failure is if $\m^n = \m^{n+1} = \dotsb$, which is impossible by Nakayama's lemma.
\end{proof}
The next step is that, because $A$ is Noetherian, the maps
\begin{equation}
\xymatrix{
	A\ar@{->>}[r]^t & \m\ar@{->>}[r]^t & \m^2\ar@{->>}[r]^t & \dotsb
}
\end{equation}
must stabilize: each is $A$ mod an ideal, and these ideals get bigger, hence must stabilize. Therefore for some
$n\gg 0$, multiplication by $t$ is an isomorphism $\m^n\to\m^{n+1}$. Now consider the pair of short exact
sequences:
\begin{equation}
\gathxy{
	0\ar[r] &\m^n\ar[r]\ar[d]^t & A\ar[r]\ar[d]^t & A/\m^n\ar[r]\ar[d]^t & 0\\
	0\ar[r] &\m^{n+1}\ar[r] & \m\ar[r] & \m/\m^{n+1}\ar[r] & 0.
}
\end{equation}
The vertical maps are all surjections; for dimensional reasons, the rightmost vertical arrow is an isomorphism, and
if $n\gg 0$, the leftmost vertical arrow is too; therefore the middle one is an isomorphism.
