Office hours are Fridays from 11-1, in room 9.164 (at least for now). Today we'll talk about some questions (and
some answers, too!) relating to algebraic geometry and why one might find it interesting. We're going to focus on
concreteness.

Broadly speaking, algebraic geometry studies zero sets of polynomials. These could be polynomials over $\Q$, or
$\R$, or $\C$, or finite fields, or more. The first question you might ask is, \emph{are there solutions}? This is
an \term{arithmetic question}: in arithmetic situations, there might not be solutions.
\begin{exm}[Taylor-Wiles, 1994]
If $n\ge 3$, the polynomial $x^n+y^n=1$ has no solutions over $\Q$ when $x,y\ne 0$.
\end{exm}
You might recognize this as a reformulation of Fermat's last theorem.

Another form of the same question is \emph{can you parameterize solutions of the equation}? For example, let's try
it with $x^2+y^2 = 1$, which we know has solutions. In this case, it is possible to parameterize solutions, via the
one-parameter family
\begin{equation}
\label{pythparam}
	x = \frac{\lambda^2-1}{\lambda^2+1}, \qquad\qquad y = \frac{2\lambda}{\lambda^2+1}.
\end{equation}
These kinds of questions are called \term{rationality questions}. One can also ask these questions over $\C$ (or
over other algebraically closed fields), where they can feel a bit different.

There is a general result that any quadric hypersurface with a rational point is rational. What this means is that
if you assume the existence of one solution $(x_0,y_0)$ to a degree-2 polynomial in $x$ and $y$ over, say, $\Q$,
then you can use that one solution to parameterize all other solutions. If you plot the solutions in the
$xy$-plane, the parameter of another solution $(x_1,y_1)$ is the slope of the line between $(x_0,y_0)$ and
$(x_1,y_1)$. Indeed, in~\eqref{pythparam}, the parameter $\lambda$ is this slope. Because the equation is a
quadric, one expects such a line to intersect in exactly two points, the first solution and another one.  This is
all extremely explicit, to the point that you could explain why you care to a middle schooler.

There are a few other rationality results.
\begin{thm}[Segre, 1940s; Manin, 1970s; Kollár, 2000]
Smooth cubics in at least three variables are rational.
\end{thm}
So $x^2+y^2 = 1$ isn't rational, but $x^2+y^2+z^2 = 1$ is. However, this doesn't give you everything.
\begin{thm}[Clemens-Griffiths, 1974]
There are cubics in at least four variables which are unirational but not rational, i.e.\ that one cannot
parameterize all solutions in a one-to-one manner.
\end{thm}
This was a hard theorem. How would you prove something like this?

Recent work (2012-15) by many people (Voisin, Colliot-Thèlene, Pirutka, Totaro\footnote{If you like pictures of
cats, check out Totaro's math blog: \url{https://burttotaro.wordpress.com/}.}) generalizes this.
\begin{thm}
For cubics in at least five variables, one can also not parameterize solutions in a one-to-one way, even by adding
additional ``dummy variables.''
\end{thm}
For four-variables cubics, this is open.
\subsection*{Schemes.}
Though this result is stated completely explicitly, it was studied using some very abstract-looking machinery. In
this course, we'll also work with this abstract machinery, namely the language of schemes. These are things like
solutions to systems of polynomials, but not quite --- they encode among other things the equivalence of such
systems under changes of coordinates, which doesn't really change the underlying geometry of the solution set.
Classification problems with this perspective are a big area of research, and Birkar just won a Fields medal for
work in this area from 2006.
\subsection*{Algebraic geometry over $\C$.}
A third thing you could care about is specific stuff about algebraic geometry over your favorite field (typically
$\C$, but not always). In many cases (such as $\C$), you have topology around, and you can ask how it interacts
with the algebraic geometry we've been talking about.

For example, if $q\in\C^\times$ isn't a root of unity, then there's a cubic equation $y^2 = x^2+ax+b$ whose
solutions are parameterized by $\C^\times/q\Z$. This may be a bit surprising, and indicates a way in which analytic
or topological information can be useful: now we can learn about the universal cover of the solution space, and
other topological invariants. Then you might ask whether something like this is true in positive characteristic,
which tends to be harder.

More generally, one can study the topology of algebraic varieties over $\C$.
\begin{thm}
The odd Betti numbers of smooth proper varieties are even.
\end{thm}
The proof uses the study of the Hodge Laplacian operator on a variety $X$. This needs a metric, but projective
means that $X$ embeds in some $\CP^n$, and we can borrow its metric. There is a purely algebro-geometric proof of
this, but first you need to come up with the right notion of Betti numbers (so étale cohomology, which is hard),
and then invoke Deligne's proof of the Weil conjectures (also hard).  Nonetheless, it's true in characteristic $p$.

More generally, the cohomology of a complex projective variety has more structure, and is much richer than that of
a random manifold.\footnote{This doesn't require smoothness \latin{per se}, but it's more difficult to formulate in
the singular case.}
\begin{conj}[Hodge conjecture, imprecise statement]
The differential topology of a projective algebraic variety over $\C$ knows everything about its algebraic
geometry.
\end{conj}
This is a Millennium Prize problem, meaning it comes with a \$1 million reward. You can infer that it's hard.
\subsection*{Algebraic geometry over $\Z$.}
If you work over $\Z$ instead of over $\C$, meaning your polynomial has integer coefficients, then you can reduce
mod $p$ and solve it there. This is the first thing anyone does in number theory, because it often simplifies the
problem to a finite question. This naturally leads one to ask, \emph{how do the systems of equations at different
primes $p$ relate to each other?}

There's a lot to say about this, beginning with quadratic reciprocity, which is very classical yet a little weird,
and continuing all the way to the Langlands program.

Supposing $X$ encodes the system of solutions to your polynomial with $\Z$ coefficients. Then one can define a zeta
function, reminiscent of the Riemann zeta function, as follows:
\begin{equation}
	\zeta_X(s)\coloneqq \prod_{\text{$p$ prime}}\exp\paren{\sum \frac{1}{n}(\text{number of solutions in
	$\F_{p^n}$}) p^{-ns}}.
\end{equation}
For $X = \Spec\Z$, corresponding to solutions to an empty set of polynomials, this recovers the usual Riemann zeta
function.

For any particular $X$, one conjectures this is meromorphic (and almost entire, in some sense), and that the
analogue of the Riemann hypothesis holds; for some $X$, this is known due to Deligne. There are some other related
conjectures related to this known as Sato-Tate conjectures.
\subsection*{Cohomology theories.}
Over $\C$, you have topology, and therefore can invoke algebraic topology to compute cohomology of algebraic
varieties. Over other fields or rings, you might not have these techniques, and there are several other
approaches.
\begin{itemize}
	\item Over an algebraically closed field, one has \term{étale cohomology}, whose ideas are built from covering
	space theory, has $\Z_\ell$ coefficients, where $\ell$ is a prime that's not the characteristic of the field.
	\item Over any field $k$, there's \term{de Rham cohomology}, which uses the idea that $\d z/z$ understands
	$\C^\times$ isn't simply connected (since $\oint \d z/z\ne 0$). This has coefficients in $k$.
\end{itemize}
There are others, too. One wants these to all be the same, or at least closely related; if $k = \Q_p$ and $\ell =
p$ ($\Q_p$ has characteristic zero!), then these two are related by $p$-adic Hodge theory. This is related to deep
and recent work by Fontaine, Scholze, and others, and relates to Scholze's Fields medal work. In 2016,
Bhatt-Morrow-Scholze showed that one can sometimes interpolate between different cohomology theories. See Scholze's
ICM address for more on this. The ultimate question in this corner of algebraic geometry is whether there's some
universal cohomology theory interpolating between everything we have, and which is also the source of the
$\zeta$-functions mentioned above.
\subsection*{Degenerations.}
We get additional power by studying solutions in families. For example, we can degenerate $x^2+y^2=1$ to
$x^2+y^2=0$, which is much simpler. One asks questions such as, \emph{what invariants are preserved under
degenerations?} Therefore one might be able to use a degeneration to reduce a harder problem to an easier problem.
\subsection*{Computations.}
This subfield of algebraic geometry tries to make these abstract invariants concrete, by writing good algorithms to
compute these invariants for explicit systems of polynomials.
\subsection*{Geometric complexity theory.}
This is another way to relate algebraic geometry and computer science. The goal of this field is to approach
another Millennium Prize problem, P vs.\ NP, using algebraic geometry techniques. This roughly involves studying
certain varieties and analyzing whether they're as complicated as they seem. Algebraic geometry has lots of
techniques which might help, but on the other hand they haven't yet.

Probably the best way to learn algebraic geometry is to have an application or research focus in mind that you can
apply the things you learn to. This method of learning tends to produce algebraic geometers.
