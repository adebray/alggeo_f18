\TODO: I may have missed stuff at the beginning. We're still proving the same theorem from last time.
\begin{lem}
Let $X\to\Spec k$ be a smooth scheme over a field $k$. If $k\inj L$ is any field extension, then $X_L\coloneqq
X\times_{\Spec k}\Spec L$ is also smooth.
\end{lem}
\begin{proof}[Proof sketch]
By Noether normalization, dimension is preserved under field extensions. If $\pi\colon X_L\to X$ is the map induced
on the pullback, then
\begin{equation}
	\Omega_{X_L/L}^1\cong \pi^*\Omega_{X/k}^1\cong \Omega^1_{X/k}\otimes_k L. \qedhere
\end{equation}
\end{proof}
Next we'll provide a useful criterion for separability.
\begin{lem}
Suppose $k\inj k'$ is a finite, inseparable field extension, and let $\overline k$ be the algebraic closure of $k$.
Then $B\coloneqq k'\otimes_k \overline k$ isn't reduced.
\end{lem}
\begin{proof}
Since $B$ is an Artinian $\overline k$-algebra, it's reduced iff it's a finite product of fields, which would mean
\begin{equation}
	B\cong \prod_{i=1}^{\dim_k k'} \overline k.
\end{equation}
Assuming this, the map $k'\to B\cong\prod \overline k$ gives $(\dim_k k')$-many distinct embeddings of $k'$ into
$k$, which means $k'$ is separable over $k$.
\end{proof}
\begin{cor}
If $\Spec k'$ isn't smooth as a scheme over $k$, then $(\Spec k')_{\overline k}$ isn't smooth over $\overline k$.
\end{cor}
In this setting, $(\Spec k')_{\overline k}$ is a product of Artinian rings which are not all fields (but their
residue fields are all $\overline k$).

Finally, now suppose $A$ is an Artinian local ring over $k$ with nonzero maximal ideal $\m$. We want to show that
$\Omega_{A/k}^1\ne 0$. We've showed it already if $A/\m$ is separable over $k$; otherwise $\Omega_{A/k}^1$ surjects
onto $\Omega_{(A/\m)/k}^1$, and we already know this is nonzero.

\orbreak

Now we'll study the local structure of smooth curves. The intuition is that, just like one-dimensional manifolds
locally look like $\R$, a smooth curve over a field $k$ will locally look like $\A_k^1$.
\begin{defn}
A \term{curve} over a field $k$ is a one-dimensional, finite type scheme over $k$.
\end{defn}
We may add more hypotheses later. For now, let's throw in the assumptions that $X = \Spec A$ is affine and smooth.

Here's a useful definition with somewhat weird notation, but everyone uses this notation, so it's worth getting
used to.
\begin{defn}
Let $X = \Spec A$ be a smooth curve and $x$ be a closed point of $X$. Then the ideal sheaf of $x$ is denoted
$\sO_X(-x)$.
\end{defn}
If $X = \Spec A$ is affine, $x = \Spec k'$ and the embedding determines a surjective map $A\surj k'$. In this case,
$\sO_X(-x)$ is the module $\sM_x\coloneqq\ker(A\surj k')$.
\begin{thm}
\label{Oxxline}
$\sO_X(-x)$ is a line bundle.
\end{thm}
\begin{rem}
Assuming \cref{Oxxline}, we can define more line bundles $\sO_X(nx)\coloneqq\sO_X(-x)^{\otimes -n}$, i.e.\
$(\sO_X(-x)^\vee)^{\otimes n}$.
\end{rem}
In particular, we have lots of line bundles on curves!

If $X$ is a smooth curve, we know $\Omega_X^1$ is a lin bundle. Locally we can trivialize it.
\begin{lem}
For every $x\in X$, there's an open $U\subset X$ containing $X$ and a map $s\colon U\to\A_k^1$ such that
$\theta_U\cdot\d s \cong \Omega_U^1$.
\end{lem}
\begin{proof}
The general setup is to suppose $\sL$ is a line bundle on $Y$, and that we have sections
$s_1,\dotsc,s_n\in\Gamma(Y,\sL)$ which define an epimorphism $\sO_Y^{\oplus n}\surj \sL$. Then $\set{s_i\ne
0}_{i=1}^n$ is an open cover of $Y$.

In our setting, we know there's a neighorhood $U_0$ containing $x$ on whiche $\Omega_X^1$ is trivial, generated say
by some $\omega = \sum f_i\ud s_i$. As in the general case, there's an $i$ such that $\set{\d s_i\ne 0}$ is an open
neighborhood of $x$. If $U = \set{\d s_i\ne 0}$, then $\d s_i$ generates $\Omega_U^1$.
\end{proof}
\begin{rem}
If you want to apply this to higher-dimensional smooth schemes, you should replace $\Omega_X^1$ with
$\Lambda^{\mathrm{top}}\Omega_X^1$.
\end{rem}
\begin{proof}[Proof of \cref{Oxxline}]
The theorem is a local statement, so we can without loss of generality find an $s\colon X\to\A_k^1$ with $\d s$
generating $\Omega_X^1$.

If $k$ is algebraically closed, then $x$ is a $k$-point of $X$, so $s(x)\in\A_k^1 = k$, which means it's a
``number,'' in that nothing weird could happen. In this case, $t\coloneqq s - s(x)\in\sO_X(-x)$ is a uniformizer:
it trivializes this line bundle in an open neighborhood of $x$.\footnote{This is the same thing as a uniformizer in
a DVR, which (\TODO) I have yet to puzzle out.}

For more general $k$, $x\in X(k')$, where $k\inj k'$ is a finite field extension. In this case we compose with the
minimal polynomial\dots
\end{proof}
