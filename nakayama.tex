We're in the middle of proving Nakayama's lemma, \cref{nakayama}. We're proving it by induction on the number of
generators of the $A$-module $M$, and the base case is trivial. So let's assume it's true for all modules generated
by $n$ elements.\footnote{The theorem is true for non-affine schemes, but we've already reduced to the affine
case.}
\begin{rem}
Let's pause to ask what a finitely generated $A$-module looks like. If it has one generator, it's isomorphic to
$A/I$ for some ideal $I$. If it has two generators, it's an extension of $A/I$ by $A/J$ for some ideals $I$ and $J$
of $A$. More generally, a module $M$ with $m$ generators is an extension $0\to N\to M\to A/I\to 0$, where $N$ has
$m-1$ generators.

% local approach!
This means some specific subcases of Nakayama's lemma, such as that for local rings, are close to
trivial.\footnote{There are many different things called Nakayama's lemma; ours is not the most general one.} You
could prove \cref{nakayama} by reducing to the local case, though we're using a different approach.

The fact that finitely generated modules have quotients which look like $A/I$ is the catalyst of the proof: it's
untrue for modules which aren't finitely generated, such as $\Q$ as a $\Z$-module, which has no quotients of the
form $\Z/n$.
\end{rem}
So $M$ is an extension of an $A$-module $N$ generated by $n$ elements by $A/I$:
\begin{equation}
\label{NMAI}
\xymatrix{
	0\ar[r] & N\ar[r] & M\ar[r] & A/I\ar[r] & 0.
}
\end{equation}
By assumption, $M\otimes_A k = 0$, which means that, since tensor product is right exact,
\begin{equation}
\label{kik}
	(A/I)\otimes_A k \cong k/Ik = 0.
\end{equation}
Recall that we had data of a map $\vp\colon A\to k$; since $k$ is a field, this and~\eqref{kik} imply there's some
$f\in I$ wth $\vp(f)\ne 0$. Let's localize at $f$; the map $\vp\colon A\to k$ passes to a map $\widetilde\vp\colon
A[f^{-1}]\to k$, and since localization is exact, \eqref{NMAI} induces a short exact sequence
\begin{equation}
	\shortexact{N[f^{-1}]}{M[f^{-1}]}{(A/I)[f^{-1}]},
\end{equation}
but since $A[f^{-1}] = 0$, $N[f^{-1}]\cong M[f^{-1}]$, which (crucially) is generated by $n$ elements as an
$A[f^{-1}]$-module. Since $\vp(f)\ne 0$, then $x\in D(f)$, so there's an open $U\subset D(f)$ containing $x$ such
that $(\sF|_{D(f)})|_U = 0$ by the inductive hypothesis, and that's exactly what we wanted to prove.
\begin{defn}
Let $M$ be an $A$-module. Then its \term{annihilator} $\Ann(M)\coloneqq\set{f\in A\mid f\cdot M = 0}$, which is an
ideal of $A$.
\end{defn}
\begin{cor}
Let $X$ be a scheme and $\sF\in\QCoh(X)$ be locally finitely generated. Then the subset $U_\sF\coloneqq\set{f\colon
\Spec B\to X\mid f^*\sF = 0}$ is an open subscheme of $X$. In particular, if $X = \Spec A$ is affine, then $U_\sF$
is the complement of the locus of $X$ on which all $f\in\Ann(\sF)$ vanish.
\end{cor}
That is, the locus where $\sF$ vanishes is open. This fits into your intuition: if you're on $\Spec A$ and $\sF$
corresponds to $A/(f)$, then $\sF$ vanishes wherever $f$ doesn't.
\begin{proof}
It suffices to prove the affine statement, and this is a matter of unwinding its definition: let $X = \Spec A$ and
$\sF$ be an $A$-module. Given $\vp\colon A\to B$, it suffices to prove the following are equivalent:
$\Ann(\sF)\cdot B = B$ and $\sF\otimes_A B = 0$.

First, the forward implication: we know there are $f_i\in\Ann(\sF)$ and $g_i\in B$ such that
\begin{equation}
	\sum_{i=1}^n \vp(f_i)g_i = 1.
\end{equation}
Therefore $1$ acts by $0$ on $\sF\otimes_A B$, so that module must be the zero module.

The reverse direction is a bit harder. Suppose for a contradiction that $\Ann(\sF)\cdot B\subsetneq B$, so it's
contained in some maximal ideal $\m$; let $k\coloneqq B/\m$, which is a field. Then 
\begin{equation}
	\sF\otimes_A k = (\sF\otimes_A B)\otimes_B k = 0.
\end{equation}
Hence, by \cref{nakayama}, there's a $U\subset\Spec A$ containing $\Spec k$ such that $\sF|_U = 0$. We can assume
$U = D(f)$ for some $f\in A$, so we're assuming $\sF[f^{-1}] = 0$. Because $\sF$ is finitely generated, this means
$f^N\sF = 0$ for some $N\gg 0$, or $f^N\in\Ann(f)$. Since $\vp(\Ann(\sF))\subset\m$, then $\vp(f^N) = 0\bmod\m$, so
$\vp(f) = 0\bmod\m$, which contradicts the assumption that $\Spec k\in U$.
\end{proof}
You can draw a picture of this: given a locally finitely generated sheaf $\sF$, $\Ann(\sF)$ has a vanishing locus;
if $\sF$ corresponds to the module $A/I$ (here we should be on an affine scheme), then this is also the closed
subset $\Spec A/I\inj\Spec A$.
\begin{ex}
Deduce every other version of Nakayama's lemma that you know (e.g.\ the one in Matsumara) from these versions.
\end{ex}
\begin{defn}
A \term{vector bundle} on a scheme $X$ is a quasicoherent sheaf $\sE\in\QCoh(X)$ which is locally finitely
generated and \term{locally projective}, i.e.\ for some (equivalently any) affine open cover $\fU$ of $X$, for
every $U = \Spec A\in\fU$, the pullback of $\sE$ to $U$ is a projective $A$-module.
\end{defn}
\begin{prop}
\label{2vb}
Let $\sE\in\QCoh(X)$. \TFAE:
\begin{enumerate}
	\item $\sE$ is a vector bundle.
	\item There is an affine open cover $\fU$ of $X$ such that for all $U\in\fU$, $\sE|_U$ is a finitely generated
	free module.
\end{enumerate}
\end{prop}
